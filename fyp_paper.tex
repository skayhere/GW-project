%preamble
\documentclass[12pt,a4paper,twocolumn]{article} 
\usepackage[lmargin=0.25in,rmargin=0.25in,tmargin=0.5in,bmargin=0.5in]{geometry}
\usepackage{times}
\usepackage{setspace}
\usepackage{graphicx}
\usepackage{subcaption}
\usepackage{amssymb}
\usepackage{amsmath}
\usepackage{ragged2e}


\renewcommand{\thesection}{\Roman{section}} 
\renewcommand{\thesubsection}{\thesection.\Roman{subsection}}
%preamble
\title{\Huge {Water table analysis using Machine Learning}}
\author{
		Aishwarya Kulkarni, Shivangi Negi, Sumedha Raghu\\		
	 \newline
	Dr. Vijaya Shetty S.\\
	\newline
	Nitte Meenakshi Institute of Technology\\	
}
\date{}
\begin{document}
		\maketitle
%\pagenumbering{roman}
\pagenumbering{arabic}


\begin{abstract}
\textbf{Groundwater is commonly the most important water resource in semi-arid areas. Monitoring water table fluctuations is essential for predicting the groundwater levels to plan for future needs. In this study, a thorough analysis is conducted concerning the prediction of groundwater levels in Karnataka. Two nonlinear data-driven models (i.e; Random Forest (RF) and gradient boosting (GB)) along with a variant of standard RNNs (Long Short-Term Memory LSTM) were proposed to predict groundwater level fluctuations. The prediction capability of these models was investigated and evaluated using yearly groundwater level data collected from observation wells located in various districts of Karnataka in India. The statistical parameters Correlation coefficient (R), Mean Square Error (MSE), precision, recall, f1 score, support, and accuracy were used to assess the performance of these models.  Different evaluation metrics highlight the capability of these models to catch the trend of groundwater level fluctuations. \\ \\ Keywords: groundwater levels; machine learning; data-driven; random forest; gradient boosting}
\end{abstract}
\begin{normalsize}
	\section{INTRODUCTION}
Groundwater is defined as the water present in the areas between the soil pore spaces and within the fractures of rock formations. The depth at which soil pore fractures and voids in rock become saturated with water is defined as the water table [1]. In many areas, groundwater has emerged as an important source of water required for domestic, irrigation, urban, and industrial activities, especially in arid and semi-arid areas. India being the largest user of groundwater in the world uses 230 cubic kilometers of groundwater per year, over a quarter of the global total [2]. For this reason, sustainable development of groundwater resources requires precise quantitative assessment and this is vital for India due to prevalent semi-arid and arid climate. Different weather conditions and usage rates are essential for the effective utilization and management of groundwater resources. Constant monitoring of the groundwater levels is important to prevent the misuse of groundwater resources that can lead to local water rationing, excessive reductions in agricultural yields, wells going dried up or producing erratic groundwater quality changes, changes in flow patterns of groundwater resulting in the inflow of poorer quality water and seawater intrusion in coastal areas [3]. The water levels, if forecasted well prior to, might facilitate the administrators to plan better, the groundwater utilization. In this research, we focus on observations wells from various districts of Karnataka. \\ \\ Traditionally, process-based models are often employed to perform groundwater simulation and predications, which rely on spatial data of the observed system dynamics. However, they are not applicable in many arid and semi-arid regions due to data limitations. On the other hand, in data-driven modeling with machine learning techniques, our model attempts to identify a direct mapping between the inputs and outputs of the system without reaching an understanding of the internal structure of the physical process [4]. The goal here is to predict groundwater levels based on temporal data inputs (historic groundwater level, weather, and rainfall data) and outputs (groundwater level). \\ \\ Recurrent neural networks (RNNs) are a popular choice for modeling groundwater time series data as they can retain a memory of past network conditions, but they face difficulty in capturing long term dependencies between variables due to exploding and vanishing gradient, where weights in the network go to zero or become extremely large during model training [5]. LSTM, a type of RNN is able to avoid these training problems by eliminating unnecessary information being passed to future model states while retaining a memory of important past events. LSTM networks have also recently been used to model the groundwater table on a monthly time step in an inland agricultural area of China [6]. This agriculture-focused study provides valuable information on the advantages of LSTM for groundwater level prediction over a basic feed-forward neural network. Groundwater level prediction is a regression problem. Based on available data we are trying to generate the best possible predictions for the groundwater level for the missing values in our dataset using binary classification. Data-driven algorithms use historical data to learn the best approximation of an underlying process. Despite the growing applications of data-driven approaches in the surface water problems, there have been only a few studies related to groundwater in arid and semi-arid regions [4]. The focus of this study is the application and comparison of two data-driven models (i.e., RF and GB) for forecasting groundwater levels in Karnataka, India.  
	
\section{LITERATURE SURVEY}	
Various works have been carried out in the field of groundwater which also include prediction analysis. The right form data extraction is highly necessary for an accurate model [4], A complex web of factors that determines groundwater levels, which are : Rainfall, aquifers, precipitation levels, seasonal changes, patterns of groundwater storage, water extraction. As highlighted in [7],[8],[9], Artificial Neural Networks (ANN) is ideal for forecasting based on implementation on data from wells and shallow aquifiers respectively. Further,different algorithms are analysed [10]. Least Squares Support Vector Machine (LSSVM) was used for dynamic forecasting of wells in Mongolia. The paper [11] helped understand the wavlet integration with support vector machine (SVM), and that it is a regression model that is being implemented to check the fluctuations in the water level. In [12], The focus is on quantitative estimates of groundwater temporally and spatially. They have performed analysis of groundwater level data in three districts of Maharashtra - Thane, Latur and Sangli. Analysis for data of more than 100 observation wells in each of these districts and developed seasonal models to represent the groundwater behavior. Three different type of models were developed-periodic, polynomial and rainfall models. In [13], a thorough analysis is conducted concerning the prediction of groundwater levels of Ljubljana polje aquifer. Three different datasets from Ljubljana (Slovenia) and Skiathos (Greece): information, pump sensor data from Skiathos and weather data. This paper processed the problem as a regression problem and hence implemented regression trees. They have highlighted the optimization when using ensembles by using random forests which is an ensemble of trees. Gradient boosting was also implemented. From the literature survey, neural networks were found to be widely used when it comes for predictive analysis. Hence for this Gradient Boosting and Random Forest are the  suitable algorithms which will be used to predict the missing values and also compare the two algorithms to see which has better accuracy. LSTM (Long short term memory) which is RNN (Recurrent neural network) model is considered for predicting the future values.



\section{SYSTEM ARCHITECTURE}
The proposed system developed using machine learning and deep learning application in which the dataset has been generated using  different sources and later analysed. The groundwater dataset is gathered from various aquifers and observational wells of several districts across Karnataka. The dataset which comprises of the multiple parameters pertaining to groundwater conditions is  split into training and testing data. Firstly cleaning of data is carried out by replacing the null values with the mean set of values. To compare and contrast the difference between a dataset with missing values and that with out one, the data is fit into the two algorithms that is Random forest and Gradient boosting. The loss functions MSE (Mean Square Error)  presented based on the models. Also output obtained are plotted which highlight the most importance of a certain parameter among all the existing parameters.Further to predict future groundwater levels, the LSTM (Long short term memory network) is applied which is a RNN (Recurrent neural network) model. These results are intended to be given to the respective authority to carry out necessary actions.\\ \\
Data preprocessing is the approach where in data is gathered from the specific account using XLS sheet. Since the gathered data is unstructured and not well-defined, the preprocessing techniques are used which include normalization, data cleaning, dimension reduction. This pre-processed data is reviewed and analysed several times by taking into consideration all of its parameters before fitting it into the model. Next the data is subjected to the prediction analyser in the form of the algorithms that includes gradient boosting, random forest and LSTM namely. Finally the outcome is compared to know how accurate is the prediction, also the result of the data cleaning and the replaced values is represnted in the form of a set of graphs. \newpage

%\begin{figure}[h]
	%\centering
%	\vspace{1cm}
%\renewcommand{\thefigure}{\thechapter.\arabic{figure}}
%	\includegraphics[scale=1]{f1}
%	\textbf{Figure 1.System Architecture}
%\end{figure}
\begin{figure}[h]
	\centering
	\vspace{1cm}
	\includegraphics[scale=0.7]{fl1}
	\caption{System Architecture}
\end{figure} 
\section{DATASET}
Table 1 represents the attributes in a given dataset. The pre-processing is done on this dataset. The dataset consists of groundwater levels in the pre-monsoon and monsoon seasons as well as the Post Rabi and Post Kharif crop seasons. The dataset also consists of location data such as the well code, its district, state, site name and site type. 
\subsection{PRE-PROCESSING}
The data collected is compiled from various government sources. The dataset initially required a significant amount of pre-processing. The techniques used on the groundwater dataset are data cleaning, data reduction and checking feature impotance.
The dataset complied contained a large amout of NaN values, due to which data was not usable. We first replaced the NaN values with 0s and then used an imputer function to replace the 0s with the mean value of each column. Certain parameters such as well code and site name were then dropped as it was not required for any of the methods employed in this project. Then a feature importance was calculated for the four features YEAR\_OBS, MONSOON, POMKH AND POMRB that contain groundwater level values. 
 \begin{figure}[h]
	\centering
	\vspace{1cm}
	\includegraphics[scale=0.5]{op13}
	\caption{Feature importance after pre-processing}
\end{figure} \newline
The plot as shown in Figure 2 depicts that POMKH (post monsoon kharif) has the highest importance. It is the top feature contributing to the predictions of the model and YEAR\_OBS (year of observation) has the lowest importance. When training a tree we can compute how much each feature contributes to decreasing the weighted impurity, which in case of regression trees is variance.
\begin{figure}[h]
\centering
\vspace{1cm}
\includegraphics[scale=0.5]{fypaper_dataset}
\textbf{Table 1.Data set description}
\end{figure} \newpage 	
\section{ALGORITHMS}
\textbf{Random Forest} : Random forest is an ensemble learning method that combines the concepts of classification and regression tasks with the use of multiple decision trees and a technique called bagging (Bootstrap Aggregation) with some additional degree of randomization.  \\ \\ Given a training set $X = x1, ..., xn$ with responses $Y = y1, ..., yn$, bagging repeatedly (B times) selects a random sample with replacement of the training set and fits trees to samples which are : \\ 
	
	For $b = 1, ..., B$ :
	\begin{itemize}
	\item Sample, with replacement, n training examples from $X$, $Y$ called $X_b$, $Y_b$. \item Train a classification or regression tree $f_b$ on $X_b$, $Y_b$. \\ \\ \end{itemize}
	Bootstrap aggregation uses the following formula to predict unseen samples by averaging prediction from individual regression trees : 
		\begin{equation}
	%\chi = \S^{	\pi \lambda}
	\hat{f} = \frac{1}{B} \sum_{b=1}^{B} {f}_{b} (\acute{x})
	\end{equation}
Where $\hat{f}$ represents the prediction for unseen samples. \\ \\
A single decision tree is a weak predictor (low bias, high variance) but is relatively fast to build. More trees give you a more robust model and prevent overfitting. When modeling, the data is resampled with replacement and for each sampling, a new classifier is trained. A new object is classified based on new attributes and each tree gives a classification. The forest chooses the classifications having the most votes of all the other trees in the forest and takes the average difference from the output of different trees. In general, different classifiers overfit the data in a different way, and through voting, those differences are averaged out. \\ \\
    \textbf{Gradient Boosting} : Gradient boosting is also based on decision trees and it builts one tree at a time. Boosting relies on weak learners (high bias, low variance) such as shallow trees, sometimes even as small as decision stumps (trees with two leaves). Here, model (ensemble) works in a forward stage-wise manner by adding one classifier at a time so that the next classifier is trained to improve the already trained ensemble, introducing a weak learner to improve the shortcomings of the existing weak learners. The gradient boosting method assumes a real-valued y and seeks an approximation $\hat{F}(x)$ in the form of a weighted sum of functions $h_i(x)$ from a class of weak learners $\gamma$.
    \begin{equation}
    \hat{F}(x) = \sum_{i=1}^{M} \gamma_{i}h_i(x)+const
    \end{equation}\\ \\
	\textbf{Long Short-term Memory Neural Networks } : LSTM neural networks are a type of RNN that was developed to overcome the vanishing and exploding gradient obstacles of traditional RNNs [5]. The LSTM architecture minimizes gradient problems by enforcing constant error flow between hidden cell states, without passing through an activation function. This study uses LSTM cells with three gates (forget gate, input gate, and output gate).
	The forward pass of LSTM networks can be described by the following equations :
	\begin{equation}
	f_t = \sigma_g(W_f x_t+U_f h_{t-1}+b_f)
    \end{equation} \begin{equation}
	i_t = \sigma_g(W_i x_t+U_i h_{t-1}+b_i)
    \end{equation}
 \begin{equation}
	o_t = \sigma_g(W_o x_t+U_o h_{t-1}+b_o)
	\end{equation}
	Where $f_t$, $i_t$ and $o_t$ can be described as forget gate, input gate and output gate respectively. The matrices $W_q$ contains the weight of the input and $U_q$ contains recurrent connections and $\sigma_g$ is the sigmoid activation function used in the LSTM network. The network output was calculated by stacking a fully connected layer on top of the LSTM cell. The product of the output layer is the forecast of the groundwater level for the coming season. 
\section{METHODOLOGY}
The groundwater dataset initially contained 14 parameters [Table 1], and these parameters had a lot of NaN values. During pre-processing, we replaced the NaN values with the value of $'0'$. Using an imputer function, we replaced the 0 valued data with the mean value calculated for each individual column. We also checked the feature importance of the parameters containing numerical values. We applied the random forest and gradient boosting algorithms on the dataset both before and after replacing the 0 valued data with the mean imputation. We also calculated the accuracies of the two algorithms before and after replacement of the null values. We divided the dataset into a 80:20 train-test split and applied the two techniques to determine accuracies, Mean squared errors, Confusion matrices and classification reports. \\ \\
For the random forest algorithm, we used a threshold value of $8.19$, calculated using the collective mean of all the groundwater values, to divide the data into $0$ and $1$ classes where any value below the threshold is classified as $0$ and anything above the threshold is classified as $1$. This division was created for the test and predicted classes to to identify the data relationship between the true-positive, true-negative, false positive and false negative. In the test split that contained $3519$ values, the following was observed.
 \begin{figure}[h]
\centering
\vspace{1cm} \includegraphics[scale=0.5]{confmatrix}  \newline \newline
\textbf{Table 2. Confusion Matrix from Random Forest Regression}
\end{figure} \newline We then calculated the accuracy and mean squared error using the random forest regression algorithm. For the gradient boosting algorithm, we made use of gradient boosting regressor function from the python libraries to calculate the accuracy and mean squared error both before and after replacing the null values. These algorithms were also used to predict the missing data in the dataset. \\ \\ 
We then implemented an RNN network, LSTM i.e. Long Short Term Memory, to predict future groundwater levels, and to check the MSE values for training and test values, where we also check the accuracy of the predicted values compared to its test and train set values. By implementing the algorithms, we yield useful and necessary results.  
\section{RESULTS} On extracting data from various sources, the data is cleaned and pre-processed, and is then fed to the random forest regressor model. To measure the effectiveness of the model the Confusion Matrix is provided. We find the recall, precision, f1-score and accuracy to have increased from $0.81$, $0.81$, $0.81$, $0.809$ to $0.85$, $0.85$, $0.85$ and $0.51$ respectively after replacing the null values with mean values. \\ \\ On implementing gradient boosting we again have two conditions i.e. with replacement of null values with mean and without replacement and we find that the MSE from $31.93$ to $19.58$ when replaced. In gradient boosting, we observe that at every iteration, we fit a base learner to the negative gradient of the loss function and multiply our prediction with a constant and add it to the value from previous iteration. From implementing the two algorithms, we observe that the MSE and Accuracy using the gradient boosting yields the best results for the dataset.  \begin{figure}[h]
	\centering
\vspace{1cm} \includegraphics[scale=0.5]{res1}  \newline \newline
\textbf{Table 3. Comparison between Random Forest Regression and Gradient Boosting} \end{figure} \newline \newline
In the Gradient Boosting algorithm, we also checked for the training set and test set deviance, where deviance is defined as a goodness-of-fit statistic for a statistical model and we found that the deviance from the actual value is very minimal as seen in Figure 3. 
\begin{figure}[h]
	\centering
	\vspace{1cm}
	\includegraphics[scale=0.7]{result2}
	\caption{Data deviation as seen in Gradient Boosting}
\end{figure} \newline
We have also implemented an RNN for prediction of future groundwater levels using LSTM algorithm, which shows the improvement in the Root Mean Squared Error (RMSE). We are able to compare the predicted values with the existing values in the dataset which is learnt by the Machine Learning model that we have used. Out of every 100 data values, we can see that the RMSE score of the LSTM model for train set compared to predicted values scored a $7.87$, whereas for the test set compared to predicted values has scored a $13.36$, which is low and beneficial. The overall performance of the models are 
found acceptable based on the high correlation efficiency.
\section{CONCLUSION}
 In this study, we propose a groundwater level forecasting system. Three data-driven methodologies are tested based on various Machine Learning algorithms; namely, random forests, gradient boosting and LSTM. The system is based on using past groundwater levels data in the pre-monsoon and monsoon seasons as well as the Post Rabi and Post Kharif crop seasons to predict the groundwater levels for the missing values in the dataset, as well as for future usage. Analysis of the results indicated that the developed gradient boosting model provided a good prediction of groundwater levels, with considerably good accuracy and lower value MSE.  The methodology and findings demonstrated in this study are useful to the research community of our nation involved in groundwater management and protection. 

\section{REFERENCES}
\justify \quad\enspace
[1] Groundwater \\
https://en.wikipedia.org/wiki/Groundwater	
\singlespacing[2] India Groundwater- World Bank Group
https://www.worldbank.org/en/news/india-groundwater-critical-diminishing \singlespacing
[3]  Thematic Papers on Groundwater-FAO
http://www.fao.org/3/a-i6040e.pdf   \singlespacing
[4] Prediction of Groundwater Level for Sustainable Water Management in an Arid Basin Using Data-driven Models
Mutao Huang and Yong Tian \singlespacing
[5] Long short-term memory (1997)
Hochreiter, S.; Schmidhuber, U. \singlespacing
[6] Developing a Long Short-Term Memory (LSTM) based model for predicting water table depth in agricultural areas.( 2018)
Zhang, J.; Zhu, Y.; Zhang, X.; Ye, M.; Yang, J. \singlespacing 
[7] Groundwater Level Predictions Using Artificial Neural Networks (Dec 2002) MAO
Xiaomin , SHANG Songhao, LIU Xiang \singlespacing
[8] Groundwater Level Forecasting in a Shallow Aquifer Using Artificial Neural Net-
work Approach Purna C. Nayak1, Y. R. Satyaji Rao1 And K. P. Sudheer2(2006) \singlespacing
[9] Application of Back-Propagation Artificial Neural Network Models for Prediction
of Groundwater Levels: Case study in Western Jilin Province, China Zhongping (2008) \singlespacing
[10] Groundwater level Dynamic prediction based on Chaos Optimization and Support
Vector Machine (2009) Jin Liu, Jian-xia Chang Wen-ge Zhang \singlespacing
[11] An integrated wavelet-support vector machine for groundwater level prediction in
Visakhapatnam,India(2014) \singlespacing
[12] Temporal Models for Groundwater Level Prediction in Regions of Maharashtra Dissertation Report -Lalit Kumar(2014) Ch. Suryanarayana a,n, Ch.Sudheer b, VazeerMaham-
mood c, B.K.Panigrahi d \singlespacing
[13] Groundwater Modeling with Machine Learning Techniques: Ljubljana polje Aquifer -
Klemen Kenda , Matej Cerin , Mark Bogataj , Matej Senožetnik , Kristina Klemen , Petra ˇ
Pergar , Chrysi Laspidou and Dunja Mladenic (2018) \singlespacing
\end{normalsize}
\end{document}